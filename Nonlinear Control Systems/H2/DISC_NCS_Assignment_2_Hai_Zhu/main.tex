\documentclass[a4 paper, 12pt]{article}
\usepackage[utf8]{inputenc}

%% preamble
% preamble.tex

%% packages
% IEEE conference needed
\usepackage{times}
\usepackage{amssymb}
\usepackage{amsmath}
% \usepackage{mathptmx}               % comment for non-IEEE style
\usepackage{graphics}
\usepackage{epsfig}
% others
\usepackage{nicefrac}               % For nice fraction like 1/2
\usepackage{geometry}               % For setting layout
\usepackage{enumitem}               % For customizing items


%% settings for normal a4 layout
% set indent
% \setlength{\parindent}{0pt}
\setlength{\parskip}{1em}
% set layout
% \geometry{a4paper,scale=0.8}
\geometry{a4paper,left=3cm,right=3cm,top=3cm,bottom=3cm}
% correct bad hyphenation here
\hyphenation{op-tical net-works semi-conduc-tor}


%% new commands
% text
\newcommand{\tn}[1]{\textnormal{#1}}
\newcommand{\tb}[1]{\textbf{#1}}
\newcommand{\ti}[1]{\textit{#1}}
% matrices
\newcommand{\mat}[0]{\begin{bmatrix}}
\newcommand{\mate}[0]{\end{bmatrix}}
% states
\newcommand{\vx}{\mathbf{x}}
\newcommand{\vp}{\mathbf{p}}
\newcommand{\vv}{\mathbf{v}}
\newcommand{\va}{\mathbf{a}}
\newcommand{\vb}{\mathbf{b}}
\newcommand{\vz}{\mathbf{z}}
\newcommand{\vd}{\mathbf{d}}
\newcommand{\vu}{\mathbf{u}}
\newcommand{\vf}{\mathbf{f}}
\newcommand{\vg}{\mathbf{g}}
\newcommand{\vh}{\mathbf{h}}
\newcommand{\vs}{\mathbf{s}}
% sets
\newcommand{\W}{\mathcal{W}}        % Workspace
\newcommand{\X}{\mathcal{X}}        % Configuration space
\newcommand{\Obs}{\mathcal{O}}      % Obstacle regions
\newcommand{\A}{\mathcal{A}}        % Robot occupied space
\newcommand{\U}{\mathcal{U}}        % Control space
\newcommand{\Z}{\mathcal{Z}}        % General decision variables space
\newcommand{\rN}{\mathcal{N}}       % Agent index sets
\newcommand{\oM}{\mathcal{M}}       % Dynamic obstacle index sets
% numbers set
\newcommand{\R}{\mathbb{R}}
\newcommand{\N}{\mathbb{N}}
% operator
\newcommand\norm[1]{\left\|#1\right\|}      % Big norm
\newcommand\abs[1]{\left|#1\right|}         % Big abs
\newcommand\inv[1]{#1^{-1}}                 % Inverse
\newcommand{\atwo}{\mathrm{atan2}}
% statistics
\newcommand{\hx}{\hat{\mathbf{x}}}
\newcommand{\hp}{\hat{\mathbf{p}}}
\newcommand{\vsigma}{\mathbf{\sigma}}
\newcommand{\vomega}{\mathbf{\omega}}
\newcommand{\mN}{\mathcal{N}}
\newcommand{\pr}{\textnormal{Pr}}
% symbols
\newcommand{\eps}{\varepsilon}
% other
\newcommand{\ra}{\rightarrow}
\newcommand{\RA}{\Rightarrow}
\newcommand{\half}{\frac{1}{2}}
\newcommand{\quater}{\frac{1}{4}}


%% theorems
\newtheorem{rmk}{\textbf{Remark}}
\newtheorem{thm}{\textbf{Theorem}}
\newtheorem{prop}{\textbf{Proposition}}
\newtheorem{prob}{\textbf{Problem}}
\newtheorem{defi}{\textbf{Definition}}
\newtheorem{algo}{\textbf{Algorithm}}
\newtheorem{cons}{\textbf{Constraint}}


%% title information
\title{
        \Large{DISC Course: Nonlinear Control Systems}\\
        \vspace{1em}
        \large\tb{Assignment 2}
}
\author{
        \small Hai Zhu                          \\
        \small Delft University of Technology   \\
        \tt\small h.zhu@tudelft.nl
 }
\date{\small\ti{Monday 19 February 2018}}

%% document
\begin{document}
%% title
\maketitle


%% Exercise 1
\tb{Exercise 1.} (ISS \& sector-bound nonlinearity)

\tb{Solution:} (a) Let $x = [x_1, x_2]^T$, then the nonlinear system can be described as follows:
\begin{equation}
        \begin{aligned}
                \dot{x_1} &= x_2 \\
                \dot{x_2} &= x_1 - 2x_2 -f(x_1) + d \\
                y &= x_1
        \end{aligned}
\end{equation}
Take the following Lyapunov function candidate
\begin{equation}
        V(x) = (x_1+x_2)^2 + 2\int_0^{x_1}f(\tau)d\tau
\end{equation} 
Then we have
\begin{equation}
        \begin{aligned}
                \dot{V}(x) &= 2x_1\dot{x_1} + 2x_2\dot{x_2} + 2\dot{x_1}x_2 + 2x_1\dot{x_2} + 2f(x_1)\dot{x_1} \\
                &= 2x_1^2 -2x_2^2 -2x_1f(x_1) + 2(x_1+x_2)d
        \end{aligned}
\end{equation}
Since $f$ is a smooth function satisfying the sector-bound condition
\begin{equation}
        c_1y^2 < f(y)y < c_2y^2
\end{equation} 
where $c_1=2$ and $c_2=6$, we can conclude that $V(x)$ is positive definite and radially bounded. That is, there exist two functions $\alpha_1,\alpha_2 \in \mathcal{K}_\infty$ such that
\begin{equation}
        \alpha_1(\norm{x}) \leq V(x) \leq \alpha_2(\norm{x})
\end{equation}
Futhermore, 
\begin{equation}
        \begin{aligned}
                \dot{V}(x) &= 2x_1^2 -2x_2^2 -2x_1f(x_1) + 2(x_1+x_2)d \\
                &< -2x_2^2 + 2(1-c_1)x_1^2 + 2(x_1+x_2)d \\
                &= -2x_1^2 -2x_2^2 + 2(x_1+x_2)d \\
                &\leq -x_1^2 -x_2^2 + 2d^2 \\
                &= -\gamma_1(\norm{x}) + \gamma_2(\norm{d})
        \end{aligned}
\end{equation}
where $\gamma_1 = r^2$ and $\gamma_2 = 2r^2$, $\gamma_1, \gamma_2 \in \mathcal{K}_\infty$. Hence, the closed-loop nonlinear system is ISS with respect to $d$.

(b) Use the same Lyapunov function as (a) and then we have
\begin{equation}
        \begin{aligned}
                \dot{V}(x) &= 2x_1^2 -2x_2^2 -2x_1f(x_1) + 2(x_1+x_2)d \\
                &< -2x_2^2 + 2(1-c_1)x_1^2 + 2(x_1+x_2)d \\
                &\leq -2x_2^2 + 2(1-c_1)x_1^2 + \eps_1 x_1^2 + \frac{d^2}{\eps_1} + \eps_2 x_2^2 + \frac{d^2}{\eps_2} \\
                &= (2-2c_1+\eps_1)x_1^2 + (\eps_2-2)x_2^2 + (\frac{1}{\eps_1}+\frac{1}{\eps_2})d^2 \\
                &\leq \max{2-2c_1+\eps_1, \eps_2-2}\norm{x}^2 + (\frac{1}{\eps_1}+\frac{1}{\eps_2})d^2
        \end{aligned}
\end{equation}
where $\eps_1, \eps_2 >0$.
Thus, to make the nonlinear system ISS with respect to $d$, the following inequalities are required
\begin{align}
        2-2c_1+\eps_1 < 0 \\
        \eps_2-2 < 0
\end{align}
Therefore, we have
\begin{equation}
        c_1 > 1+\frac{\eps_1}{2} > 1
\end{equation}
Hence, the lower bound of $c_1$ is $1$.


%% Exercise 2
\tb{Exercise 2.} (Counter-examples)

\tb{Solution:} (a) Consider a scalar dynamical system
\begin{equation}
        \dot{x} = -x + x^2u
\end{equation}
where $x\in\R$ and $x\neq c$, $c$ is some constant. If $u=0$, then system becomes $\dot{x} = -x$. Take $V(x) = \half x^2$ the Lyapunov function candidate, then we have $\dot{V}(x)=-x^2$, which is negative definite. Therefore, the system is globally asymptotically stable with zero input.

However, if we take a bounded signal $u \equiv 1$, we can find the solution of the system as follows
\begin{equation}
        x = \frac{1}{1-e^{t-\ln{\frac{x(0)}{x(0)-1}}}}
\end{equation}
where $x(0)$ is the initial condition. Since $x(t)$ is not a constant, it is obvious that $x(0) \neq 1$ in the above equation. Hence, the system with the above solution has a finite-escape time $\ln{\frac{x(0)}{x(0)-1}}$.

(b) Consider the following state equation
\begin{align}
        \dot{x}_1 &= x_1 + x_2 + u \\
        \dot{x}_2 &= -x_2 + x_2^2d
\end{align}
If $d = 0$, we can choose $u = -2x_1-x_2$ as the state feedback law. Then the state equation becomes
\begin{align}
        \dot{x}_1 &= -x_1\\
        \dot{x}_2 &= -x_2
\end{align}
Take $V(x) = \half x_1^2 + \half x_2^2$ the Lyapunov function candidate, then we have $\dot{V}(x) = x_1\dot{x}_1 + x_2\dot{x}_2 = -x_1^2-x_2^2$,  which is negative definite. Therefore, the system is globally asymptotically stable with zero disturbance and the state feedback law.

However, according to the results in (a), if $d \equiv 1$, then $x_2(t)$ is unbounded and has a finite escape time. Futhermore, note that the state feedback law $u=\alpha(x_1,x_2)$ cam only affect $x_1$. Thus, no matter what state feedback law is chosen, $x_2$ is unbounded and hence, the system cannot be made ISS respect to the disturbance signal $d$.


%% Bibliography
% \bibliographystyle{plain}
% \begin{thebibliography}{99}

%         \bibitem{c1} H.K. Khalil. \ti{Nonlinear systems}. Prentice Hall, Upper Saddle River, USA, third edition, 2002.

%         \bibitem{c2} C.T. Chen. \ti{Linear system theory and design}. Oxford University Press, Inc., USA, 1998.

    
% \end{thebibliography}


\end{document}
