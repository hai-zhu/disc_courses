\documentclass[a4 paper, 12pt]{article}
\usepackage[utf8]{inputenc}

%% preamble
% preamble.tex

%% packages
% IEEE conference needed
\usepackage{times}
\usepackage{amssymb}
\usepackage{amsmath}
% \usepackage{mathptmx}               % comment for non-IEEE style
\usepackage{graphics}
\usepackage{epsfig}
% others
\usepackage{nicefrac}               % For nice fraction like 1/2
\usepackage{geometry}               % For setting layout
\usepackage{enumitem}               % For customizing items
% For tables
\usepackage{multirow}
\usepackage{makecell}
\usepackage{diagbox}

\usepackage{enumitem}

\usepackage{amsthm}

\usepackage{cite}



%% settings for normal a4 layout
% set indent
% \setlength{\parindent}{0pt}
\setlength{\parskip}{1em}
% set layout
% \geometry{a4paper,scale=0.8}
\geometry{a4paper,left=3cm,right=3cm,top=3cm,bottom=3cm}
% correct bad hyphenation here
\hyphenation{op-tical net-works semi-conduc-tor}


%% new commands
% text
\newcommand{\tn}[1]{\textnormal{#1}}
\newcommand{\tb}[1]{\textbf{#1}}
\newcommand{\ti}[1]{\textit{#1}}
% matrices
\newcommand{\mat}[0]{\begin{bmatrix}}
\newcommand{\mate}[0]{\end{bmatrix}}
% states
\newcommand{\vx}{\mathbf{x}}
\newcommand{\vp}{\mathbf{p}}
\newcommand{\vv}{\mathbf{v}}
\newcommand{\va}{\mathbf{a}}
\newcommand{\vb}{\mathbf{b}}
\newcommand{\vz}{\mathbf{z}}
\newcommand{\vd}{\mathbf{d}}
\newcommand{\vu}{\mathbf{u}}
\newcommand{\vf}{\mathbf{f}}
\newcommand{\vg}{\mathbf{g}}
\newcommand{\vh}{\mathbf{h}}
\newcommand{\vs}{\mathbf{s}}
% sets
\newcommand{\W}{\mathcal{W}}        % Workspace
\newcommand{\X}{\mathcal{X}}        % Configuration space
\newcommand{\Obs}{\mathcal{O}}      % Obstacle regions
\newcommand{\A}{\mathcal{A}}        % Robot occupied space
\newcommand{\U}{\mathcal{U}}        % Control space
\newcommand{\Z}{\mathcal{Z}}        % General decision variables space
\newcommand{\rN}{\mathcal{N}}       % Agent index sets
\newcommand{\oM}{\mathcal{M}}       % Dynamic obstacle index sets
% numbers set
\newcommand{\R}{\mathbb{R}}
\newcommand{\N}{\mathbb{N}}
% operator
\newcommand\norm[1]{\left\|#1\right\|}      % Big norm
\newcommand\abs[1]{\left|#1\right|}         % Big abs
\newcommand\inv[1]{#1^{-1}}                 % Inverse
\newcommand{\atwo}{\mathrm{atan2}}
\newcommand\mc[1]{\mathcal{#1}}
% statistics
\newcommand{\hx}{\hat{\mathbf{x}}}
\newcommand{\hp}{\hat{\mathbf{p}}}
\newcommand{\vsigma}{\mathbf{\sigma}}
\newcommand{\vomega}{\mathbf{\omega}}
\newcommand{\mN}{\mathcal{N}}
\newcommand{\pr}{\textnormal{Pr}}
% symbols
\newcommand{\eps}{\varepsilon}
% other
\newcommand{\ra}{\rightarrow}
\newcommand{\RA}{\Rightarrow}
\newcommand{\half}{\frac{1}{2}}
\newcommand{\quater}{\frac{1}{4}}


%% theorems
\newtheorem{rmk}{\textbf{Remark}}
\newtheorem{thm}{\textbf{Theorem}}
\newtheorem{prop}{\textbf{Proposition}}
\newtheorem{prob}{\textbf{Problem}}
\newtheorem{defi}{\textbf{Definition}}
\newtheorem{algo}{\textbf{Algorithm}}
\newtheorem{cons}{\textbf{Constraint}}


%% title information
\title{
        \Large{DISC Course: Nonlinear Control Systems}\\
        \vspace{1em}
        \large\tb{Assignment 3}
}
\author{
        \small Hai Zhu                          \\
        \small Delft University of Technology   \\
        \tt\small h.zhu@tudelft.nl
 }
\date{\small\ti{Monday 26 February 2018}}

%% document
\begin{document}
%% title
\maketitle


%% Exercise 1
\tb{Exercise 1} (13.12 from \cite{c1})

\tb{Solution:}
(a) The system dynamics can be written in the form $\dot{x} = f(x) + g(x)u$ where $x = [x_1,x_2,x_3]^T$ and 
\begin{align}
        f(x) &= \mat -x_1+x_1x_2 \\ x_2+x_3 \\ \delta(x) \mate \\
        g(x) &= \mat 0 \\ 0 \\ 1 \mate
\end{align}
Let $ h(x) = y = x_1 + x_2$, then the system can be described as follows
\begin{align}
        y &= x_1 + x_2 \\
        \dot{y} &= \dot{x}_1 + \dot{x}_2 = -x_1+x_1x_2+x_2+x_3 \\
        \ddot{y} &= -\dot{x}_1 + \dot{x}_1x_2 + x_1\dot{x}_2 + \dot{x}_2 + \dot{x}_3 \\
        &= (-1+x_2)(-x_1+x_1x_2) +(x_1+1)(x_2+x_3)+\delta(x)+u
\end{align}
It is obvious that the system has a relative degree of 2 in $\R^3$ ($L_gL_fh(x) = 1$). Hence, the system is input-output linearizable.

(b) Consider the following change of variables
\begin{align}
        \eta &= \phi(x) \\
        \xi_1 &= h(x) = x_1+x_2 \\
        \xi_2 &= L_fh(x) = -x_1+x_1x_2+x_2+x_3
\end{align}
where $\phi(x)$ satisfies
\begin{equation}
        L_g\phi(x) = \frac{\partial\phi(x)}{\partial x}g(x) = 0
\end{equation}
which gives a solution $\phi(x) = x_1$. Thus, consider the following coordinate transformation matrix
\begin{equation}
        T(x) = \mat \eta \\ \xi_1 \\ \xi_2 \mate = \mat x_1 \\ x_1+x_2 \\ -x_1+x_1x_2+x_2+x_3 \mate
\end{equation}
which is a diffeomorphism in the domain $\{ x\in\R^3 | x\neq0 \}$. Furthermore, it transforms the system into the following normal form 
\begin{align}
        \dot\eta &= -\eta + \eta(\xi_1-\eta) \\
        \dot\xi_1 &= \xi_2 \\
        \dot\xi_2 &= (-1+x_2)(-x_1+x_1x_2) +(x_1+1)(x_2+x_3)+\delta(x)+u
\end{align}

(c) Consider the zero dynamics of the system $y(t) = 0$, which results in $\xi_1(t)=0$ and $\xi_2(t)=0$. Therefore, the dynamics of $\eta$ is described as follows
\begin{equation}
        \dot\eta = -\eta - \eta^2
\end{equation}
Let $\eta(0) = \eta_0 \neq -1$, then we can obtain the solution of the above differential equation
\begin{equation}
        \eta(t) = -\frac{e^{C-t}}{e^{C-t}-1}, \quad C = \log\frac{y_0}{y_0+1}
\end{equation}
Thus, $\lim_{t\ra\infty}\eta(t) = 1$. Therefore, the origin of the zero dynamics is asymptotically stable. Hence, the system is minimum phase.

(d) According to the system dynamics, we have
\begin{equation}
        ad_fg(x) = -\mat -1+x_2 &x_1 &0 \\ 0 &1 &1 \\ \frac{\partial{\delta(x)}}{\partial{x_1}} &\frac{\partial{\delta(x)}}{\partial{x_2}} &\frac{\partial{\delta(x)}}{\partial{x_3}} \mate \mat 0 \\ 1 \\ 1 \mate = \mat 0 \\ -1 \\ -\frac{\partial{\delta(x)}}{\partial{x_3}} \mate
\end{equation}
\begin{equation}
        \begin{aligned}
                ad_f^2g(x) = &-\mat 0 &0 &0 \\ 0 &0 &0 \\ \frac{\partial^2{\delta(x)}}{\partial{x_1^2}} &\frac{\partial^2{\delta(x)}}{\partial{x_2^2}} &\frac{\partial^2{\delta(x)}}{\partial{x_3^2}} \mate \mat -x_1+x_1x_2 \\ x_2+x_3 \\ \delta(x) \mate \\
                 &-\mat -1+x_2 &x_1 &0 \\ 0 &1 &1 \\ \frac{\partial{\delta(x)}}{\partial{x_1}} &\frac{\partial{\delta(x)}}{\partial{x_2}} &\frac{\partial{\delta(x)}}{\partial{x_3}} \mate \mat 0 \\ -1 \\ -\frac{\partial{\delta(x)}}{\partial{x_3}}  \mate 
                 = \mat x_1 \\ \star \\ \star \mate
        \end{aligned}
\end{equation}
Consider the matrix $M = [g(x) ~~ad_fg(x) ~~ad_f^2g(x)]$
\begin{equation}
        M = \mat 0 &0 &x_1 \\ 0 &-1 &\star \\ 1 &-\frac{\partial{\delta(x)}}{\partial{x_3}} &\star \mate
\end{equation}
whose rank is 3 if $x_1\neq0$ and we have $\det{(M)}=x_1$. Hence, the system is feedback linearizable in the domain $\{ x\in\R^3 | x\neq0 \}$.

(e) The objective is to find an output $h(x)$ which satisfies 
\begin{align}
        h(0) &= 0 \\
        L_gh(x) &= \frac{\partial{h(x)}}{\partial{x}}g(x) = 0 \\
        L_gL_fh(x) &= \frac{\partial{L_fh(x)}}{\partial{x}}g(x) = 0 \\
        L_gL_f^2h(x) &= \frac{\partial{L_f^2h(x)}}{\partial{x}}g(x) \neq 0
\end{align} 
It can be verified that $h(x) = x_1$ is a solution of the above equations, which results in 
\begin{align}
        L_fh(x) &= \frac{\partial{h(x)}}{\partial{x}}f(x) = -x_1+x_1x_2 \\
        L_f^2h(x) &= \frac{\partial{L_fh(x)}}{\partial{x}}f(x) = x_1(-1+x_2)^2 + x_1(x_2+x_3)
\end{align}
Thus, consider the following transformation matrix
\begin{equation}
        T(x) = [h(x), L_fh(x), L_f^2h(x)]^T = \mat x_1 \\ -x_1+x_1x_2  \\ x_1(-1+x_2)^2 + x_1(x_2+x_3) \mate
\end{equation}
which is a diffeomorphism in the domain $\{ x\in\R^3 | x\neq0 \}$. Taking $z=T(x)$, the resulting transformed system is
\begin{align}
        \dot{z}_1 &= z_2 \\
        \dot{z}_2 &= z_3 \\
        \dot{z}_3 &=  x_1(u-\alpha(x))
\end{align}
and the feedback control law is taken as $u = \alpha(x) + \frac{v}{x_1}$.


%% Exercise 2
\tb{Exercise 2} (13.17 from \cite{c1})

\tb{Solution:} 
The system dynamics can be written in the form $\dot{x} = f(x) + g(x)u$ where $x = [x_1,x_2,x_3]^T$ and 
\begin{align}
        f(x) &= \mat -x_1+x_2 \\ x_1-x_2-x_1x_3 \\ x_1+x_1x_2-2x_3 \mate \\
        g(x) &= \mat 0 \\ 1 \\ 0 \mate
\end{align}
Thus, we can calculate
\begin{equation}
        ad_fg(x) = [f(x), g(x)] = \mat -1 \\ 1 \\ -x_1 \mate \\
\end{equation}
\begin{equation}
        [g(x), ad_fg(x)] = [0 ~0 ~0]^T
\end{equation}
and 
\begin{equation}
        ad_f^2g(x) = [f(x), ad_fg(x)] = \mat -2 \\ 2-x_3-x_1^2 \\ 1-2x_1 \mate
\end{equation}
Consider the matrix $M = [g(x) ~~ad_fg(x) ~~ad_f^2g(x)]$
\begin{equation}
        M = \mat 0 &-1 &-2 \\ 1 &1 &2-x_3-x_1^2 \\ 0 &-x_1 &1-2x_1 \mate
\end{equation}
where $\det(M)=1$. Therefore, the system is feedback linearizable. The objective is to find an output $h(x)$ which satisfies 
\begin{align}
        h(0) &= 0 \\
        L_gh(x) &= \frac{\partial{h(x)}}{\partial{x}}g(x) = 0 \\
        L_gL_fh(x) &= \frac{\partial{L_fh(x)}}{\partial{x}}g(x) = 0 \\
        L_gL_f^2h(x) &= \frac{\partial{L_f^2h(x)}}{\partial{x}}g(x) \neq 0
\end{align} 
It can be verified that $h(x) = x_1^2 - 2x_3$ is a solution of the above equations, which results in 
\begin{align}
        L_fh(x) &= \frac{\partial{h(x)}}{\partial{x}}f(x) = -2(x_1^2+x_1-2x_3) \\
        L_f^2h(x) &= \frac{\partial{L_fh(x)}}{\partial{x}}f(x) = 4x_1^2+6x_1-2x_2-8x_3
\end{align}
Thus, consider the following transformation matrix
\begin{equation}
        T(x) = [h(x), L_fh(x), L_f^2h(x)]^T = \mat x_1^2 - 2x_3 \\ -2(x_1^2+x_1-2x_3) \\ 4x_1^2+6x_1-2x_2-8x_3 \mate
\end{equation}
which can be verified is a global diffeomorphism. Taking $z=T(x)$, we can obtain
\begin{align}
        \dot{z}_1 &= z_2 \\
        \dot{z}_2 &= z_3 \\
        \dot{z}_3 &= -8x_1^2 +2x_1x_3 - 16x_1 + 8x_2 + 16x_3 + 2u \\
                  &= -4z_2 - 4z_3 + 2(u + x_1x_3)
\end{align}
Thus, the above equations can be written in the following form
\begin{equation}
        \dot{z} = Az - 2B(u-\alpha(x))
\end{equation}
Hence, the feedback control law to globally stabilize the origin can be taken as $u=\alpha(x)+\half K$, where $K$ is a matrix such that $A-BK$ is Hurwitz.


%% Exercise 3
\tb{Exercise 3} (13.24 from \cite{c1})

\tb{Solution:} 
Since $V(\eta,\xi) = V_0(\eta) + \lambda\sqrt{\xi^TP\xi}$, we have
\begin{equation}
        \begin{aligned}
                \dot{V}(\eta,\xi) &= \frac{V_0(\eta)}{\partial{\eta}}\dot\eta + \frac{\lambda}{2\sqrt{\xi^TP\xi}}(\dot\xi^TP\xi + \xi^TP\dot\xi) \\
        \end{aligned}
\end{equation}
Note that $P$ satisfies 
\begin{equation}
        P(A-BK) + (A-BK)^TP = -I
\end{equation}
from which we can conclude that $P$ is symmetric. 
Hence, we have 
\begin{equation}
        \begin{aligned}
                \dot{V}(\eta,\xi) &= \frac{V_0(\eta)}{\partial{\eta}}f(\eta,\xi) +  \frac{\lambda}{2\sqrt{\xi^TP\xi}}(-\xi^T\xi + 2\xi^TPB\delta(z)) \\
                &= \frac{V_0(\eta)}{\partial{\eta}}f(\eta,0) + \frac{V_0(\eta)}{\partial{\eta}}[f(\eta,\xi) - f(\eta,0)] +  \frac{\lambda}{2\sqrt{\xi^TP\xi}}(-\xi^T\xi + 2\xi^TPB\delta(z)) \\
        \end{aligned}
\end{equation}
Note that $f_0(\eta,\xi)$ is local Lipschitz and $\frac{V_0(\eta)}{\partial{\eta}}$ is bounded in some neighborhood of $\eta=0$. We have 
\begin{equation}
        \frac{V_0(\eta)}{\partial{\eta}}[f(\eta,\xi) - f(\eta,0)] \leq c_1\norm{\xi}
\end{equation}
where $c_1$ is a positive constant.
Furthermore, since 
\begin{equation}
        \frac{V_0(\eta)}{\partial{\eta}}f(\eta,0) \leq -W(\eta)
\end{equation}
and 
\begin{equation}
       \norm{\delta(z)} \leq k(\norm{\xi} + W(\eta)) 
\end{equation}
We have
\begin{equation}
        \begin{aligned}
                \dot{V}(\eta,\xi) &= \frac{V_0(\eta)}{\partial{\eta}}f(\eta,0) + \frac{V_0(\eta)}{\partial{\eta}}[f(\eta,\xi) - f(\eta,0)] +  \frac{\lambda}{2\sqrt{\xi^TP\xi}}(-\xi^T\xi + 2\xi^TPB\delta(z)) \\
                &\leq -W(\eta) + c_1\norm{\xi} -c_2\lambda\norm{\xi} + c_3\lambda\norm{\delta}  \\ 
                &\leq -W(\eta) + c_1\norm{\xi} -c_2\lambda\norm{\xi} + kc_3\lambda\norm{\xi} + kc_3\lambda W(\eta) \\
                &= -(1-kc_3\lambda)W(\eta) - (c_2\lambda - c_1 - kc_3\lambda)\norm{\xi}
        \end{aligned}
\end{equation}
where $c_1, c_2, c_3$ are positive constants.

Now, in the Lyapunov function candidate $V(\eta,\xi)$, we take $\lambda = 2c_1/c_2$ and let $c^\star = \frac{\min\{1,c_1\}}{2c_3\lambda}$. Then, for $\forall k<c^\star$ we have
\begin{equation}
        \begin{aligned}
                \dot{V}(\eta,\xi) &\leq -(1-kc_3\lambda)W(\eta) - (c_2\lambda - c_1 - kc_3\lambda)\norm{\xi} \\
                &\leq -\half W(\eta) - \half c_1\norm{\xi}
        \end{aligned}
\end{equation}
Hence, the origin $z=0$ is asymptotically stable for sufficiently small $k$.


%% Bibliography
\bibliographystyle{plain}
\begin{thebibliography}{99}

        \bibitem{c1} H.K. Khalil. \ti{Nonlinear systems}. Prentice Hall, Upper Saddle River, USA, third edition, 2002.

            
\end{thebibliography}


\end{document}
