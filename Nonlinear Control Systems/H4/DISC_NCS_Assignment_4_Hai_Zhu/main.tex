\documentclass[a4 paper, 12pt]{article}
\usepackage[utf8]{inputenc}

%% preamble
% preamble.tex

%% packages
% IEEE conference needed
\usepackage{times}
\usepackage{amssymb}
\usepackage{amsmath}
% \usepackage{mathptmx}               % comment for non-IEEE style
\usepackage{graphics}
\usepackage{epsfig}
% others
\usepackage{nicefrac}               % For nice fraction like 1/2
\usepackage{geometry}               % For setting layout
\usepackage{enumitem}               % For customizing items


%% settings for normal a4 layout
% set indent
% \setlength{\parindent}{0pt}
\setlength{\parskip}{1em}
% set layout
% \geometry{a4paper,scale=0.8}
\geometry{a4paper,left=3cm,right=3cm,top=3cm,bottom=3cm}
% correct bad hyphenation here
\hyphenation{op-tical net-works semi-conduc-tor}


%% new commands
% text
\newcommand{\tn}[1]{\textnormal{#1}}
\newcommand{\tb}[1]{\textbf{#1}}
\newcommand{\ti}[1]{\textit{#1}}
% matrices
\newcommand{\mat}[0]{\begin{bmatrix}}
\newcommand{\mate}[0]{\end{bmatrix}}
% states
\newcommand{\vx}{\mathbf{x}}
\newcommand{\vp}{\mathbf{p}}
\newcommand{\vv}{\mathbf{v}}
\newcommand{\va}{\mathbf{a}}
\newcommand{\vb}{\mathbf{b}}
\newcommand{\vz}{\mathbf{z}}
\newcommand{\vd}{\mathbf{d}}
\newcommand{\vu}{\mathbf{u}}
\newcommand{\vf}{\mathbf{f}}
\newcommand{\vg}{\mathbf{g}}
\newcommand{\vh}{\mathbf{h}}
\newcommand{\vs}{\mathbf{s}}
% sets
\newcommand{\W}{\mathcal{W}}        % Workspace
\newcommand{\X}{\mathcal{X}}        % Configuration space
\newcommand{\Obs}{\mathcal{O}}      % Obstacle regions
\newcommand{\A}{\mathcal{A}}        % Robot occupied space
\newcommand{\U}{\mathcal{U}}        % Control space
\newcommand{\Z}{\mathcal{Z}}        % General decision variables space
\newcommand{\rN}{\mathcal{N}}       % Agent index sets
\newcommand{\oM}{\mathcal{M}}       % Dynamic obstacle index sets
% numbers set
\newcommand{\R}{\mathbb{R}}
\newcommand{\N}{\mathbb{N}}
% operator
\newcommand\norm[1]{\left\|#1\right\|}      % Big norm
\newcommand\abs[1]{\left|#1\right|}         % Big abs
\newcommand\inv[1]{#1^{-1}}                 % Inverse
\newcommand{\atwo}{\mathrm{atan2}}
% statistics
\newcommand{\hx}{\hat{\mathbf{x}}}
\newcommand{\hp}{\hat{\mathbf{p}}}
\newcommand{\vsigma}{\mathbf{\sigma}}
\newcommand{\vomega}{\mathbf{\omega}}
\newcommand{\mN}{\mathcal{N}}
\newcommand{\pr}{\textnormal{Pr}}
% symbols
\newcommand{\eps}{\varepsilon}
% other
\newcommand{\ra}{\rightarrow}
\newcommand{\RA}{\Rightarrow}
\newcommand{\half}{\frac{1}{2}}
\newcommand{\quater}{\frac{1}{4}}


%% theorems
\newtheorem{rmk}{\textbf{Remark}}
\newtheorem{thm}{\textbf{Theorem}}
\newtheorem{prop}{\textbf{Proposition}}
\newtheorem{prob}{\textbf{Problem}}
\newtheorem{defi}{\textbf{Definition}}
\newtheorem{algo}{\textbf{Algorithm}}
\newtheorem{cons}{\textbf{Constraint}}


%% title information
\title{
        \Large{DISC Course: Nonlinear Control Systems}\\
        \vspace{1em}
        \large\tb{Assignment 4}
}
\author{
        \small Hai Zhu                          \\
        \small Delft University of Technology   \\
        \tt\small h.zhu@tudelft.nl
 }
\date{\small\ti{\today}}

%% document
\begin{document}
%% title
\maketitle


%% Exercise 1
\tb{Exercise 1} 

\tb{Solution:} (a) Let $x = [x_1, x_2]^T = [q, \dot{q}]^T$, then we have
\begin{align}
        \dot{x}_1 &= x_2 \label{eq:1.x1} \\ 
        \dot{x_2} &= \frac{1}{m}(-dx_2 - g + u) \label{eq:1.x2}
\end{align}
where $m = m(x_1)$, $d = d(x_1,x_2)$, $g = g(x_1)$ are functions describing the mass, damping and spring. The above equation can be written in the following state-space form
\begin{align}
        \dx &= f(x) + k(x)u = \mat x_2 \\ -\frac{1}{m(x_1)}(d(x_1,x_2)x_2+g(x_1)) \mate + \mat 0 \\ -\frac{1}{m(x_1)} \mate u \\
        y &= h(x) = \mat 1 &0 \mate x
\end{align}

Lie derivatives
\begin{align}
        \mathcal{L}_fh(x) &= x_2 \\
        \mathcal{L}_kh(x) &= 0 \\
        \mathcal{L}_f^2h(x) &= -\frac{1}{m(x_1)}(d(x_1,x_2)x_2+g(x_1)) \\
        \mathcal{L}_k\mathcal{L}_fh(x) &= -\frac{1}{m(x_1)}
\end{align}
Therefore, the system has uniform relative degree 2. Let 
\begin{align}
        \xi_1 &= h(x) = x_1 \\
        \xi_2 &= \mathcal{L}_fh(x) = x_2
\end{align}
Find a coordinate $\eta_1 = \phi_1(x)$ such that $\mathcal{L}_k\phi_1(x)=0$, i.e., 
\begin{equation}
        \frac{\partial\phi_1}{\partial x}(x)k(x) = -\frac{\partial\phi_1}{\partial x_2}\frac{1}{m(x_1)} = 0
\end{equation}
A solution is $\eta_1 = \phi_1(x) = x_1$. Coordinate transformation
\begin{equation}
        T(x) = \mat \xi_1 \\ \xi_2 \\ \eta \mate = \mat h(x) \\ L_fh(x) \\ \phi_1(x) \mate = \mat x_1 \\ x_2 \\ x_1 \mate
\end{equation}
Normal form
\begin{align}
        \dot{\xi}_1 &= \xi_2 \\
        \dot{\xi}_2 &= -\frac{1}{m(\xi_1)}(d(\xi_1,\xi_2)+g(\xi_1)) + \frac{1}{m(\xi_1)}u \\
        \dot{\eta_1} &= \xi_1
\end{align}


(b) In equation (\ref{eq:1.x1}), let 
\begin{equation}
        x_2 = -x_1 \defeq k_1(x_1)
\end{equation}
Take $V_1(x_1) = \half x_1^2$ as the Lyapunov function candidate, then $\dot{V_1}(x_1) = x_1\dx_1 = -x_1^2$. Thus, the feedback control law stabilizes the system described by equation (\ref{eq:1.x1}). 

Let $w_1 = x_2 - k_1(x_1)$, then 
\begin{align}
        \dx_1 &= x_2 = w_1 + k_1(x_1) = -x_1 + w_1 \\
        \dot{w}_1 &= \dx_2 - \dot{k}_1(x_1) = -\frac{1}{m(x_1)}(d(x_1,x_2)x_2+g(x_1)) + \frac{1}{m(x_1)}u + x_2
\end{align}
Take $V(x) = V(x_1,w_1) = V_1(x_1) + \half w_1^2 = \half x_1^2 + \half w_1^2$, then 
\begin{equation}
        \begin{aligned}
                \dot{V}(x_1,w_1) &= x_1\dx_1 + w_1\dot{w}_1 \\
                &= x_1(-x_1+w_1) + w_1(-\frac{1}{m(x_1)}(d(x_1,x_2)x_2+g(x_1))+\frac{1}{m(x_1)}u+x_2) \\
                &= -x_1^2 + w_1(x_1-\frac{1}{m(x_1)}(d(x_1,x_2)x_2+g(x_1))+\frac{1}{m(x_1)}u+x_2)
        \end{aligned}
\end{equation}
Let 
\begin{equation}\label{eq:1.u}
        \begin{aligned}
                u &= m(x_1)(-x_2 + \frac{1}{m(x_1)}(d(x_1,x_2)x_2+g(x_1)) -x_1 - c_1w_1) \\
                &= -m(x_1)(c_1+1)(x_1+x_2) + d(x_1,x_2)x_2 + g(x_1) \\
                &\defeq \alpha(x)
        \end{aligned}
\end{equation}
where $c_1 > 0$. Thus
\begin{equation}
        \dot{V}(x_1,w_1) = -x_1^2 - c_1w_1^2 \leq 0
\end{equation}
Therefore, the feedback law (\ref{eq:1.u}) stabilizes the origin of the nonlinear system.


%% Exercise 2
\tb{Exercise 2} 

\tb{Solution:} (a) The nonlinear system is described as
\begin{align}
        \dx_1 &= x_1^3 + x_2 \label{eq:2.dx1} \\
        \dx_2 &= x_2 + x_1 + u \label{eq:2.dx2}
\end{align}
In equation (\ref{eq:2.dx1}), Let
\begin{equation}\label{eq:2.k1}
        x_2 = -2x_1^3 \defeq k_1(x_1)
\end{equation}
Take $V_1(x_1) = \quater x_1^4$ as the Lyapunov function, then
\begin{equation}
        \begin{aligned}
                \dot{V}_1(x_1) &= x_1^3\dx_1 = x_1^3(x_1^3-2x_1^3) = -x_1^6 \leq 0
        \end{aligned}
\end{equation}
Thus, the feedback law (\ref{eq:2.k1}) stabilize the system (\ref{eq:2.dx1}).

Let $w_1 = x_2 - k_1(x_1) = x_2 + 2x_1^3$, then we have
\begin{align}
        \dx_1 &= x_1^3+x_2 = -x_1^3 + w_1 \\
        \dot{w}_1 &= \dx_2+6x_1^2\dx_1 = -6x_1^5 + 6x_1^2w_1 + x_2 + x_1 + u
\end{align}
Take $V(x) = V(x_1,w_1) = V_1(x_1) + \half w_1^2 = \quater x_1^4 + \half w_1^2$ as the Lyapunov function candidate, then
\begin{equation}
        \begin{aligned}
                \dot{V}(x_1,w_1) &= x_1^3\dx_1 + \dot{w}_1w_1 \\
                &= x_1^3(-x_1^3+w_1) + w_1(-6x_1^5 + 6x_1^2w_1 + x_2 + x_1 + u) \\
                &= -x_1^6 + w_1(x_1^3 - 6x_1^5 + 6x_1^2w_1 + x_1 + x_2 + u)
        \end{aligned}
\end{equation}
Let 
\begin{equation} \label{eq:2.u}
        \begin{aligned}
                u &= -x_1^3 + 6x_1^5 - 6x_1^2w_1 - x_1 - x_2 - w_1 \\
                &= -6x_1^5 - 3x_1^3 - 6x_1^2x_2 - x_1 -2x_2 \\
                &\defeq \alpha(x)
        \end{aligned}
\end{equation}
then
\begin{equation}
        \dot{V}(x_1,w_1) = -x_1^6 - w_1^2 \leq 0
\end{equation}
which implies that feedback law (\ref{eq:2.u}) stabilizes the given nonlinear system.

(b) Let $d = [d_1,d_2]^T, x = [x_1, w_1]^T$, considering the uncertainty in the system and base on the results in (a), we have
\begin{equation}
        \begin{aligned}
                \dot{V}(x_1,w_1) &= x_1^3\dx_1 + \dot{w}_1w_1 \\
                &= -x_1^6 - w_1^2 + x_1^3d_1 + w_1(d_2 + 6x_1^2d_1) \\
                &= -x_1^6 - w_1^2 + x_1^3d_1 + w_1d_2 + 6x_1^2w_1d_1 \\
                &\leq -x_1^6 - w_1^2 + (\frac{x_1^6}{2} + \frac{d_1^2}{2}) + (\frac{w_1^2}{4} + d_2^2) + (\frac{w_1^2}{2} + \frac{x_1^6}{3} + 6^5d_1^6) \\
                &= -\frac{1}{6}x_1^6 - \frac{1}{4}w_1^2 + (\half+6^5)d_1^2 + d_2^2 \\
                &\leq -\gamma_1(\norm{x}) + \gamma_2(\norm{d})
        \end{aligned}
\end{equation}
where $\gamma_1(z) = \frac{1}{6}z^6$, $\gamma_2(z) = (\half+6^5)z^2 $
and $\gamma_1, \gamma_2 \in \mathcal{K}_\infty $. 

Hence, with the feedback control law $u = \alpha(x)$, the closed-loop system is ISS with respect to $d_1$ and $d_2$.


%% Bibliography
% \bibliographystyle{plain}
% \begin{thebibliography}{99}

%         \bibitem{c1} H.K. Khalil. \ti{Nonlinear systems}. Prentice Hall, Upper Saddle River, USA, third edition, 2002.

%         \bibitem{c2} C.T. Chen. \ti{Linear system theory and design}. Oxford University Press, Inc., USA, 1998.

    
% \end{thebibliography}


\end{document}
