\documentclass[a4 paper, 12pt]{article}
\usepackage[utf8]{inputenc}

%% preamble
% preamble.tex

%% packages
% IEEE conference needed
\usepackage{times}
\usepackage{amssymb}
\usepackage{amsmath}
% \usepackage{mathptmx}               % comment for non-IEEE style
\usepackage{graphics}
\usepackage{epsfig}
% others
\usepackage{nicefrac}               % For nice fraction like 1/2
\usepackage{geometry}               % For setting layout
\usepackage{enumitem}               % For customizing items
% For tables
\usepackage{multirow}
\usepackage{makecell}
\usepackage{diagbox}

\usepackage{enumitem}

\usepackage{amsthm}

\usepackage{cite}



%% settings for normal a4 layout
% set indent
% \setlength{\parindent}{0pt}
\setlength{\parskip}{1em}
% set layout
% \geometry{a4paper,scale=0.8}
\geometry{a4paper,left=3cm,right=3cm,top=3cm,bottom=3cm}
% correct bad hyphenation here
\hyphenation{op-tical net-works semi-conduc-tor}


%% new commands
% text
\newcommand{\tn}[1]{\textnormal{#1}}
\newcommand{\tb}[1]{\textbf{#1}}
\newcommand{\ti}[1]{\textit{#1}}
% matrices
\newcommand{\mat}[0]{\begin{bmatrix}}
\newcommand{\mate}[0]{\end{bmatrix}}
% states
\newcommand{\vx}{\mathbf{x}}
\newcommand{\vp}{\mathbf{p}}
\newcommand{\vv}{\mathbf{v}}
\newcommand{\va}{\mathbf{a}}
\newcommand{\vb}{\mathbf{b}}
\newcommand{\vz}{\mathbf{z}}
\newcommand{\vd}{\mathbf{d}}
\newcommand{\vu}{\mathbf{u}}
\newcommand{\vf}{\mathbf{f}}
\newcommand{\vg}{\mathbf{g}}
\newcommand{\vh}{\mathbf{h}}
\newcommand{\vs}{\mathbf{s}}
% sets
\newcommand{\W}{\mathcal{W}}        % Workspace
\newcommand{\X}{\mathcal{X}}        % Configuration space
\newcommand{\Obs}{\mathcal{O}}      % Obstacle regions
\newcommand{\A}{\mathcal{A}}        % Robot occupied space
\newcommand{\U}{\mathcal{U}}        % Control space
\newcommand{\Z}{\mathcal{Z}}        % General decision variables space
\newcommand{\rN}{\mathcal{N}}       % Agent index sets
\newcommand{\oM}{\mathcal{M}}       % Dynamic obstacle index sets
% numbers set
\newcommand{\R}{\mathbb{R}}
\newcommand{\N}{\mathbb{N}}
% operator
\newcommand\norm[1]{\left\|#1\right\|}      % Big norm
\newcommand\abs[1]{\left|#1\right|}         % Big abs
\newcommand\inv[1]{#1^{-1}}                 % Inverse
\newcommand{\atwo}{\mathrm{atan2}}
\newcommand\mc[1]{\mathcal{#1}}
% statistics
\newcommand{\hx}{\hat{\mathbf{x}}}
\newcommand{\hp}{\hat{\mathbf{p}}}
\newcommand{\vsigma}{\mathbf{\sigma}}
\newcommand{\vomega}{\mathbf{\omega}}
\newcommand{\mN}{\mathcal{N}}
\newcommand{\pr}{\textnormal{Pr}}
% symbols
\newcommand{\eps}{\varepsilon}
% other
\newcommand{\ra}{\rightarrow}
\newcommand{\RA}{\Rightarrow}
\newcommand{\half}{\frac{1}{2}}
\newcommand{\quater}{\frac{1}{4}}


%% theorems
\newtheorem{rmk}{\textbf{Remark}}
\newtheorem{thm}{\textbf{Theorem}}
\newtheorem{prop}{\textbf{Proposition}}
\newtheorem{prob}{\textbf{Problem}}
\newtheorem{defi}{\textbf{Definition}}
\newtheorem{algo}{\textbf{Algorithm}}
\newtheorem{cons}{\textbf{Constraint}}


%% title information
\title{
        \Large{DISC Course: Multi-agent Network Dynamics and Games}\\
        \vspace{1em}
        \large\tb{Homework 3}
}
\author{
        \small Hai Zhu                          \\
        \small Delft University of Technology   \\
        \tt\small h.zhu@tudelft.nl
 }
\date{\small\ti{\today}}

%% document
\begin{document}
%% title
\maketitle

% \section{Homework 3}
% ~~~~
\tb{Problem 1}

\tb{Solution: }
\tb{a)} For different values of $x$,
\begin{itemize}
        \item Case 1: $x=0$. \\
        The payoff matrix is 
        \begin{center}
                \begin{tabular}{ | m{1em} | m{1em}| } 
                \hline
                1,1& 2,0 \\ 
                \hline
                0,2 & 3,3 \\ 
                \hline
                \end{tabular}
        \end{center}
        There are two pure strategy Nash equilibria: $(e_1,e_1)$ and $(e_2,e_2)$.

        Apparently, $(e_2,e_2)$ is a strict symmetric Nash equilibrium. Thus, $e_2$ is an evolutionary stable strategy (Corollary 5.53 in \cite{b1}). For $e_1$, since we have $\pi(e_1,e_1)>\pi(e_2,e_1)$, thus $e_1$ is also an evolutionary stable strategy.

        \item Case 2: $x=1$. \\
        The payoff matrix is 
        \begin{center}
                \begin{tabular}{ | m{1em} | m{1em}| } 
                \hline
                1,1& 2,1 \\ 
                \hline
                1,2 & 3,3 \\ 
                \hline
                \end{tabular}
        \end{center}
        There are two pure strategy Nash equilibria: $(e_1,e_1)$ and $(e_2,e_2)$.

        Similarly, since $(e_2,e_2)$ is a strict symmetric Nash equilibrium. Thus, $e_2$ is an evolutionary stable strategy (Corollary 5.53 in \cite{b1}). For $e_1$, we have $\pi(e_1,e_1)=\pi(e_2,e_1)$ but $\pi(e_1,e_2)<\pi(e_2,e_2)$. So $e_1$ is not an evolutionary stable strategy.

        \item Case 3: $x = 2$. \\
        The payoff matrix is 
        \begin{center}
                \begin{tabular}{ | m{1em} | m{1em}| } 
                \hline
                1,1& 2,2 \\ 
                \hline
                2,2 & 3,3 \\ 
                \hline
                \end{tabular}
        \end{center}
        There is only one pure strategy Nash equilibria: $(e_2,e_2)$. 

        Note that the only one pure strategy Nash equilibrium is strict symmetric. Thus $e_2$ is the evolutionary stable strategy (Corollary 5.53 in \cite{b1}). 

\end{itemize}

\tb{b)} Recall the definition of weakly dominated strategy \cite{b1}:
\begin{defi}
        A strategy $s_i$ of player $i$ is termed weakly dominated if there exists another strategy $t_i$ of player $i$ satisfying the following two conditions:\vspace{-0.5em}

        (i) For every strategy vector $s_{-i}$ of other players,
        \begin{equation}\label{eq:WD1}
                \pi(s_i,s_{-i}) \leq \pi(t_i,s_{-i})
        \end{equation}
        \vspace{-2.5em}

        (ii) There exists a strategy vector $t_{-i}$ of other players such that 
        \begin{equation}\label{eq:WD2}
                \pi(s_i,t_{-i}) < \pi(s_i,t_{-i})
        \end{equation}
\end{defi}
If the equal condition in equation (\ref{eq:WD1}) is removed, then the above definition is refined to ``strictly dominated''.

In the given game, since $X$ is weakly dominated and $(X,X)$ is a Nash equilibrium, we have 
\begin{align}
        & c = a \\
        & d > b
\end{align}
We can compute
\begin{equation}
        \pi(X,X) = a
\end{equation}
\begin{equation}
        \pi(Y,X) = c
\end{equation}
We further compute that 
\begin{equation}
        \pi(X,Y) = b
\end{equation}
\begin{equation}
        \pi(Y,Y) = d
\end{equation}
Thus we have $\pi(X,X) = \pi(Y,X)$ but $\pi(X,Y) < \pi(Y,Y)$. Hence, $X$ is not an evolutionary stable strategy.



\tb{Problem 2}

\tb{Solution:} \tb{a)} 
Let $s_1 = e_1$ the first pure strategy of player 1. We can suppose $t_i = e_2, e_3, e_4$ and check if it satisfies the above conditions. The result they all do not satisfy the conditions since 
\begin{equation}
        \begin{aligned}
                \pi(e_1,e_2) > \pi(e_2,e_2) \\
                \pi(e_1,e_1) > \pi(e_3,e_1) \\
                \pi(e_1,e_3) > \pi(e_4,e_3)
        \end{aligned}
\end{equation}
which contradicts with equation (\ref{eq:WD1}). Hence, the first pure strategy of player 1 is not dominated by a pure strategy.

\tb{b)} We suppose that it is weakly dominated by a mixed strategy $p = [p_1,p_2,p_3,p_4]^T$. Then the payoff of player 1 by choosing this mixed strategy is 
\begin{equation}
        \pi(p,q) = p^TAq
\end{equation}
where $q = [q_1,q_2,q_3,q_4]$ is a strategy of the other player. Then according to the definition of ``weakly dominated'', we can get the following condition
\begin{align}
        &p^TAq \geq e_1^TAq, \hspace{0.2cm} \forall q\in Q \\
        &p^TAq > e_1^Tq, \hspace{0.2cm} \exists q\in Q
\end{align}
where $Q$ is the strategy space of the other player. The above conditions can be written more clearly as follows
\begin{align}
        &[p_1,p_2,p_3,p_4]A \geq [1,2,0,-2] \\
        &\norm{[p_1,p_2,p_3,p_4]A} > \norm{[1,2,0,-2]} \\
        &p_1 + p_2 + p_3 + p_4 = 1
\end{align}
To solve the above underdetermined equation, we can use optimization based method such as linear programming. Here is a solution $p = [0, \frac{2}{3}, \frac{1}{3}, 0]^T$. Furthermore, we can valid that this mixed strategy actually strictly dominants the first pure strategy.

\tb{c)} Yes. We have shown that the mixed strategy $p = [0, \frac{2}{3}, \frac{1}{3}, 0]^T$ strictly dominants the first pure strategy in previous question.

\tb{d)} No, it is impossible. According to the Theorem 5.20 in \cite{b1}, in every Nash equilibrium of a game in strategic form, the pure strategy strictly dominated by a mixed strategy is chosen by the player with probability 0. Since we have shown that the first pure strategy of player 1 is strictly dominated by a mixed strategy, it is impossible that it is at a Nash equilibrium.

\tb{e)} Use the definition of strictly dominated, we can valid the following statements:
\begin{itemize}
        \item The first pure strategic $e_1$ of player 1 is strictly dominated by a mixed strategy $p = [0, \frac{2}{3}, \frac{1}{3}, 0]^T$.
        \item The forth pure strategic $e_4$ of player 1 is strictly dominated by a mixed strategy $p = [0, \frac{2}{3}, \frac{1}{3}, 0]^T$.
\end{itemize}
Therefore, the set of Nash equilibrium is confined to the space of strategies $e_2$ and $e_3$, esulting in the following two-player payoff table:
\begin{center}
        \begin{tabular}{ | m{1em} | m{1em}| } 
        \hline
        1,1& 1,4 \\ 
        \hline
        4,1 & 3,3 \\ 
        \hline
        \end{tabular}
\end{center}
It can be verified that 
\begin{align}
        \pi(e_3,e_3) > \pi(e_2,e_3) \label{eq:2.Na1} \\
        \pi(e_3,e_3) > \pi(e_3,e_2) \label{eq:2.Na2}
\end{align}
Hence, $(e_3,e_3)$ is a pure strategy Nash equilibrium of the symmetric game.

\tb{f)} According to the Theorem 5.51 in \cite{b1}, for any two-player symmetric game, an evolutionary stable equilibrium is also a Nash equilibrium. Since we have shown that there is only one Nash equilibrium $(e_3,e_3)$ of the given game in previous questions, we only need to check if it is evolutionary stable. 

Please observe equation (\ref{eq:2.Na1}) and (\ref{eq:2.Na2}). It shows that $(e_3,e_3)$ is a strict symmetric Nash equilibrium, then the conditions for ESS hold (Corollary 5.53 \cite{b1}). Hence, $e_3$ is an evolutionary stable strategy in this game.


\tb{Problem 3}
\begin{proof}

To show that ``can be'', I only need to give an example to support the statement. Consider the following payoff matrix
\begin{equation}
        \left[
        \begin{array}{ccc}
        0 & 1 & 0 \\
        0 & 0 & 2 \\
        0 & 0 & 1
        \end{array}
        \right]
\end{equation}
The replicator dynamics are
\begin{equation}
        \left[
        \begin{array}{c}
        \dot{x_1} \\
        \dot{x_2} \\
        \dot{x_3} 
        \end{array}
        \right]
        =
        \left[
        \begin{array}{c}
        -x_1(x_1x_2 - x_2 + x_3(2x_2 + x_3))\\
        -x_2(x_1x_2 - 2x_3 + x_3(2x_2 + x_3))\\
        -x_3(x_1x_2 - x_3 + x_3(x_2 + x_3))
        \end{array}
        \right]
\end{equation}

A $\omega$-limit point is a point where the dynamics are zero. For the replicator dynamics (RP) means that $\pi(e^i,x) = \pi(x,x)$. Furthermore, if $\hat{x}\in \Delta$ and is a $\omega$-limit point then $\hat{x} \in \Delta^{NE}$ (Preposition 3.5 in \cite{b2}). 

It can be verified that in this game, $(e_1,e_1)$ is the unique Nash equilibrium. However, $e_1$ is not Lyapunov stable. The proof is given below: Define a ball around $e^1$ with $r>0$ as $B_r=\{x\in \Delta | \hspace{0.5em}||x-e_1|| \le r\}$. Furthermore, let $V(x)=\frac{1}{2}^Tx$ be a Lyapunov function candidate. Then we have $\dot{V(x)}=x^T\dot{x}$. According to the Theorem 4.3 in \cite{b5} for $\forall x^* \in B_r$ if $\dot{V(x)}>0$ then $e^1$ is unstable. If we check for the neighbor mixed strategy of $e^1$ $x_{test}=[1-\epsilon \hspace{0.5em} \epsilon \hspace{0.5em}0]^T | \epsilon >0$ we get $\dot{V(x_{test})=
e^2*(2*e^2 - 3*e + 1)>0}$ with $\epsilon$ sufficiently small. So $e_1$ is unstable. Futhermore, since all $\omega$-limit point of all orbits x(t) in the interior of simplex converge to $e_1$ because it is the strongly dominant strategy. Hence, The state is unstable. This completes the proof.

\end{proof}


\tb{Problem 4}
\begin{proof}

Recall the definition of asymptotically stable set (Theorem 6.3 \cite{b2}):
\begin{defi}
        Suppose that $A \in C$ is a closed set. Then $A$ is asymptotically stable if and only if there exists a neighborhood $D$ of $A$ and a continuous function $v:D\ra R_+$ satisfying the following conditions:
        \begin{equation}
                v(x) = 0 \hspace{0.4cm} \tn{if and only if } x\in A,
        \end{equation}
        \begin{equation}
                v(\xi(t,x)) < v(x) \hspace{0.4cm} \tn{if } x\notin A, t>0, \tn{and } \xi(s,x)\in D ~~\forall s\in[0,t].
        \end{equation}
\end{defi}

For this theorem, there is a proof in \cite{b2} (Proposition 3.13). The main idea goes as follows:
        
Consider that $X \subset \delta$ is a evolutionary stable set (ES set). Then $\forall x \in X$, there exists $W_x$ be a neighbor of $x$ such that:

\begin{equation}
\pi(x,y) > \pi(y,y)
\end{equation}

$\forall y \in W_x \setminus\{x\}$. Let $Q_x \in \Delta$ be the set of mixed strategies $y \in \Delta$ that assign positive probabilities to all pure strategies with positive probabilities assigned by $x$:

\begin{equation}
Q_x=\{y \in \Delta : C(x) \subset C(y)\}
\end{equation}

Then $V_x=W_x \cap Q_x $  is a (relative) neighborhood of $x$ on which the entropy function $H_x$ is defined. After, we identify a neighborhood $P$ of $X$ which is a basin of attraction for $X$. Hence, $\forall x \in \Delta \exists \alpha_x \in \mathbb{R} : \alpha_x > 0$ such that the lower contour set $P_x=\{y \in Q_x : H_x(y) < \alpha_x\}$ is contained in the above neighborhood $V_x$. Let $P$ be the union of all $P_x$. Then $P \subset \Delta$ is a neighborhood of $X$ (relative to $\Delta$). Furthermore, if $y \in P\setminus\{x\} $, then $y \in P_x$ for some $x \in X$, and $\dot{H_x}(y<0)$ for each such x.

For each $y \in P$ let $X(y)=\{x \in X : C(x) \subset C(y) \}$ with the function $H$ defined as:

\begin{equation}
H(y)=min(H_x(y))
\end{equation}

By Berge's maximum theorem, $H$ is continuous. Furthermore, $H(y) \ge 0 \forall y \in P$ if and only if $y \in X$. As we defined previously  $y \in P\setminus\{x\} $, consequently $\xi(x_0,t)\in P$ and $H(\xi(x_0,t))<H(x_0 \forall t > 0)$ Hence, X is asymptotically stable.


\end{proof}





%% Bibliography
\bibliographystyle{plain}
\begin{thebibliography}{99}

        \bibitem{b1} M. Maschler, E. Solan, and S. Zamir. \ti{Game Theory.} Cambridge: Cambridge University Press, 2013.

        \bibitem{b2} Weibull, Jörgen W. \ti{Evolutionary Game Theory}. MIT Press, 1997.

        \bibitem{b3} W. H. Sandholm, E. Dokumaci, and F. Franchetti. \ti{Dynamo: Diagrams for Evolutionary Game Dynamics}. 2012. http://www.ssc.wisc.edu/~whs/dynamo.

        % \bibitem{b4} Riehl, J. R., and Cao, M. Minimal-agent control of evolutionary games on tree networks. In The 21st International Symposium on Mathematical Theory of Networks and Systems (Vol. 148), 2014.

        \bibitem{b5} H.K. Khalil. \ti{Nonlinear systems}. Prentice Hall, Upper Saddle River, USA, third edition, 2002.

\end{thebibliography}


\end{document}
